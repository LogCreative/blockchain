\documentclass{sjtuarticle}
\subject{区块链技术}
\title{第四次作业}
\author{李子龙\\123033910195}
\usepackage{multicol}
\usepackage{ntheorem}
\usepackage{listings}
\lstset{aboveskip = 0pt,
belowskip = 0pt,
basicstyle=\ttfamily,mathescape=true}
\usepackage{hyperref}
\begin{document}
\maketitle

\begin{solution}
A = Alice ($PK_1,SK_1$), B = Bob ($PK_2,SK_2$), C = Carol ($PK_3,SK_3$), D = David ($PK_4,SK_4$), E = Eve ($PK_5,SK_5$)

\begin{multicols}{2}
\begin{description}
    \item[$TX_1$]:
    
        \begin{description}
            \item[Input]:
                \begin{description}
                    \item[prev]: $H(TX_0)$
                    \item[n]: 0 
                    \item[scriptSig]:
\begin{lstlisting}
Sign($SK_4$,$TX_0$)
$PK_4$
\end{lstlisting}
                \end{description}
            \item[Output]:
                \begin{description}
                    \item[{$TXO[0]$}]:
                    \begin{description}
                        \item[value]: 100
                        \item[scriptPubKey]:
\begin{lstlisting}
OP_DUP
OP_HASH160
$H(PK_4)$
OP_EQUALVERIFY
OP_CHECKSIG
\end{lstlisting}
                    \end{description}
                    \item[{$TXO[1]$}]:
                    \begin{description}
                        \item[value]: 100
                        \item[scriptPubKey]:
\begin{lstlisting}
OP_2
$H(PK_1)$
$H(PK_2)$
$H(PK_3)$
OP_3
OP_CHECKMULTISIG
\end{lstlisting}
                    \end{description}
                \end{description}    
        \end{description}
\columnbreak
    \item[$TX_2$]:
        \begin{description}
            \item[Input]:
            \begin{description}
                \item[prev]: $H(TX_1)$
                \item[n]: 1 
                \item[scriptSig]:
\begin{lstlisting}
OP_0
Sign($SK_1$,$TX_1$)
Sign($SK_2$,$TX_1$)
\end{lstlisting}
            \end{description}
            \item[Output]:
            \begin{description}
                \item[{$TXO[0]$}]:
                \begin{description}
                    \item[value]: 60
                    \item[scriptPubKey]:
\begin{lstlisting}
OP_DUP
OP_HASH160
$H(PK_5)$
OP_EQUALVERIFY
OP_CHECKSIG
\end{lstlisting}
                \end{description}
                \item[{$TXO[1]$}]:
                \begin{description}
                    \item[value]: 40
                    \item[scriptPubKey]:
\begin{lstlisting}
OP_2
$H(PK_1)$
$H(PK_2)$
$H(PK_3)$
OP_3
OP_CHECKMULTISIG
\end{lstlisting}
                \end{description}
            \end{description}
        \end{description}
\end{description}
\end{multicols}

这里假设 $TX_1$ 中 D 给 ABC 公司的 TXO 在 [1] 位置上; $TX_2$ 中提供了 A 和 B 的签名,给 E 的 TXO 在 [0] 位置上。这里假设没有交易费。

其中 \verb"OP_CHECKMULTISIG" 的前置参数是
\begin{lstlisting}
OP_0 <Sig_1> ... <Sig_$M$> OP_$M$ <PubKeyHash_1> ... <PubKeyHash_$N$> OP_$N$
\end{lstlisting}
只需要提供 $N$ 个公钥对应的 $M$ 个签名即可通过 \verb"OP_CHECKMULTISIG"。\verb"OP_0" 是占位符(无操作),\verb"OP_1" ~ \verb"OP_16" 输出对应数字。

参考文献:\url{https://en.bitcoin.it/wiki/Script}

\end{solution}

\end{document}