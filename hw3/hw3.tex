\documentclass{sjtuarticle}
\subject{区块链技术}
\title{第三次作业}
\author{李子龙\\123033910195}
\usepackage{ntheorem}
\begin{document}
\maketitle
\begin{enumerate}
    \item \begin{solution}
        在挑战阶段,已经得到 $m_b^*$($b$ 从 $\{0,1\}$ 中随机选取)的密文 $c^*$。在 El Gammal 加密中,该密文具有这种形式:$c^*=(c_1^*,c_2^*)$。在检测阶段,不允许直接对 $c^*$ 进行解密。为了能够得到 $m_b^*$,将要求对新构造的 $c=(c_1^*,c_1^*\cdot c_2^*)$ 进行解密,得到 $m$ 后,就可以得到 $m_b^*=(c_1^*)^{-1}\cdot m$,此时与 $m_0$ 与 $m_1$ 进行对比就可以确定 $b$,概率会显著大于 $1/2$,也就意味着 El Gammal 加密不是 CCA 安全的。

        证明如下:根据 El Gammal 加密的定义,对于公钥 $\mathit{pk}=(G,q,g,h)$,私钥 $\mathit{sk}=(x)$,其中$h=g^x$,对于消息 $m_b\in G$ 有密文 $c^*=(c_1^*,c_2^*)=(g^y,m_b\cdot h^y)$。根据群的封闭性,$c_1^*\in G,c_2^*\in G, c_1^*\cdot c_2^*\in G$,故构造的密文 $c=(c_1^*,c_1^*\cdot c_2^*)\in G^2$ 合法。解密时,有
        \begin{equation*}
            m = c_1^* \cdot c_2^*\cdot ({c_1^*}^x)^{-1}=c_1^*\cdot m_b\cdot h^y\cdot (g^{xy})^{-1} = c_1^* \cdot m_b\cdot h^y\cdot (h^y)^{-1}=c_1^*\cdot m_b
        \end{equation*}
        则等式两边与 $(c_1^*)^{-1}$ 进行左侧二元运算, 有
        \begin{equation*}
            (c_1^*)^{-1}\cdot m = (c_1^*)^{-1}\cdot c_1^*\cdot m_b = m_b
        \end{equation*}
        %由于在 El Gammal 加密中,$({c_1^*}^x)^{-1}$ 被认为是可求的,所以
        可以认为 $(c_1^*)^{-1}=(c_1^*)^{q-1}$ 是可求的,证毕。
    \end{solution}
    \item \begin{proof}
        \begin{enumerate}
            \item 一个散列函数是碰撞抵抗的 $\Rightarrow$ 目标碰撞抵抗的:所有的多项式算法中,已知碰撞抵抗,即对于 $\forall (x,x^\prime)\in \{0,1\}^{*2}$ 且 $x\neq x^\prime$,此处认为序列对 $(x,x^\prime)$ 是有序的输出,即 $(x,x^\prime)\neq (x^\prime,x)$,都有
            \begin{equation}\label{eq:cr}
                P\left(H(x)=H(x^\prime)\right)\leq \text{negl}(n)
            \end{equation}
            记 $P(H(x_i)=H(x^\prime))$ 为固定一个定义域中的 $x_i\in \{0,1\}^*$,选取 $x^\prime\in \{0,1\}^*$ 且 $x^\prime\neq x_i$ 使得散列结果相同的概率,有下面的关系
            \begin{equation}\label{eq:crtcr}
                P\left(H(x)=H(x^\prime)\right)=\sum_{i} P(H(x_i)=H(x^\prime))
            \end{equation}
            结合式 \eqref{eq:cr} 和式 \eqref{eq:crtcr},以及概率的定义 $0\leq P\leq 1$,有
            \begin{equation}
                P(H(x_i)=H(x^\prime)) \leq \text{negl}(n)
            \end{equation}
            也就是它也是目标碰撞抵抗的。
            \item 一个散列函数是目标碰撞抵抗的 $\Rightarrow$ 原像抵抗的:反证法。假设所有的多项式算法中,对于固定的原像 $y\in \{0,1\}^{l(n)}$,找到 $x\in \{0,1\}^*$ 使得 $H(x)=y$ 的概率不再是可忽略的,即
            \begin{equation}\label{eq:a1}
                P(H(x)=y)>\text{negl}(n)
            \end{equation}
            由于值域空间是小于定义域空间的,$|\{0,1\}^{l(n)}|<|\{0,1\}^*|$,实际上后者是一个无穷大的空间,所以对于一个原像 $y$,存在定义域中的另一个解 $x^\prime\in \{0,1\}^*$ 且 $x^\prime\neq x$,使得 $y=H(x^\prime)$。
            
            就原像抵抗的试验而言,$x$ 和 $x^\prime$ 的地位是等价的、不可区分的,所以记
            \begin{equation}\label{eq:a2}
                p=P(H(x^\prime)=y)=P(H(x)=y)>\text{negl}(n)
            \end{equation}

            现在做两次原像抵抗试验,假设第一次试验成功得到 $x$ 使得 $H(x)=y$;第二次试验也成功,得到 $x^\prime$ 使得 $H(x^\prime)=y$,有两种情形:

            \begin{enumerate}
                \item $x^\prime=x$,第二次与第一次取到的值相同;
                \item $x^\prime\neq x$,第二次与第一次取到的值不相同。
            \end{enumerate}
            显然
            \begin{equation}\label{eq:one}
                P(x^\prime=x)+P(x^\prime\neq x)=1
            \end{equation}

            以目标碰撞抵抗的视角而言,可以视作第一次试验得到的 $x$ 是固定的,即在 $H(x)=y$ 的条件下,第二次试验成功的概率被分解为
            \begin{equation}\label{eq:decomp}
                P(H(x^\prime)=y)=P(x^\prime=x)P\left(H(x^\prime)=y|x^\prime=x\right)+P(x^\prime\neq x)P\left(H(x^\prime)=y|x^\prime\neq x\right)
            \end{equation}
            注意到因为定义域空间 $|\{0,1\}^*|$ 是无穷大的,第二次取值直接为第一次取值的概率极限为0,结合式 \eqref{eq:one} 有
            \begin{align*}
                P(x^\prime=x)&=0\\
                P(x^\prime\neq x)&=1-P(x^\prime=x)=1
            \end{align*}
            
            那么式 \eqref{eq:decomp} 就变为
            \begin{equation}
                P(H(x^\prime)=y)=P(H(x^\prime)=y|x^\prime\neq x)
            \end{equation}

            结合式 \eqref{eq:a2} 有
            \begin{equation}\label{eq:a3}
                P(H(x^\prime)=y|x^\prime\neq x)=p>\text{negl}(n)
            \end{equation}

            式 \eqref{eq:a3} 左边正是在 $x^\prime\neq x$ 的条件下,目标碰撞抵抗 $H(x^\prime)=y$ 成功的概率,该式表明这个概率是不可忽略的。这与目标碰撞抵抗成功的概率可忽略的前提是矛盾的。
        \end{enumerate}
    \end{proof}
\end{enumerate}
\end{document}