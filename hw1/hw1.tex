\documentclass{sjtuarticle}
\subject{区块链技术}
\title{第一次作业}
\author{李子龙\\123033910195}
\begin{document}
\maketitle
\begin{enumerate}
    \item 必答题
    \begin{enumerate}
        \item 解:计算$\gcd (227,79)$如下:
        \begin{align*}
            227&=79*2+69\\
            79&=69*1+10\\
            69&=10*6+9\\
            10&=9*1+1\\
            9&=1*9+0
        \end{align*}
        所以 $\gcd(227,79)=1$,即227和79互质。
        \item 答:没有。由于7932和11958都是偶数,至少有一个公因数2,所以两个数不互质,不满足含有模逆的前提条件。
        \item 解:根据欧拉函数的定义,$\phi(21)=21\times\left(1-\frac{1}{3}\right)\left(1-\frac{1}{7}\right)=12$,则
        \begin{align*}
            227^{54996213} \bmod 21 &= 227^{54996213\bmod \phi(21)} \bmod 21\\ 
            &= 227^{54996213\bmod 12} \bmod 21 \\
            &= 227^{9} \bmod 21
        \end{align*}
        由于 $227^{9}=227^8\times 227^1$,所以使用平方相乘法:
        \begin{align*}
            227\bmod 21 &= 17 & 227^2\bmod 21 &= 17^2\bmod 21 = 15 \\
            227^4\bmod 21 &= 15^2 \bmod 21 = 15 & 227^8\bmod 21 &= 15^2 \bmod 21 = 15
        \end{align*}
        故
        \begin{align*}
            227^{54996213} \bmod 21 &= 227^{9} \bmod 21 = (227^8\times 227^1 ) \bmod 21 \\
            &= ((227^8 \bmod 21) \times (227 \bmod 21))\bmod 21\\
            &= (15\times 17)\bmod 21 \\
            &= 3
        \end{align*}
        \item 解:由于质因数分解 $730=2\times 5\times 73$,根据欧拉函数的定义:
        \begin{align*}
            \phi(730) &= 730\times \left(1-\frac{1}{2}\right)\left(1-\frac{1}{5}\right)\left(1-\frac{1}{73}\right)\\
            &= 288
        \end{align*}
    \end{enumerate}
    \item 选答题\\
        解:使用扩展欧拉算法
        \begin{align*}
            229 &= 2\times 79 + 71 \\
            79 &= 1\times 71 + 8\\
            71&= 8\times 8 + 7 \\
            8&= 1\times 7+1
        \end{align*}
        反向,
        \begin{align*}
            1&=8-1\times 7 \\
            &=8-1\times(71-8\times 8)=-1\times 71+9\times 8\\
            &=-1\times 71+9\times (79-1\times 71) = 9\times 79-10\times 71\\
            &=9\times 79-10\times (229-2\times 79)=-10\times 229+29\times 79
        \end{align*}
        也就是
        \begin{equation*}
            \gcd(79,229)=1=-10\times 229+29\times 79
        \end{equation*}
        两侧同余于 229,
        \begin{equation*}
            1 = 29\times 79\bmod 229
        \end{equation*}
        也就是 $29$ 为 $79\bmod 229$ 的模逆。
\end{enumerate}
\end{document}