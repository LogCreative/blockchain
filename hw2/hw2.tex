\documentclass{sjtuarticle}
\subject{区块链技术}
\title{第二次作业}
\author{Log Creative}
\date{2023 年 9 月 26 日}
\begin{document}
\maketitle
\begin{enumerate}
    \item 
    \begin{enumerate}
        \item 解:是。$Z_5=\{0,1,2,3,4\}$,则
        \begin{description}
            \item[封闭性] 若 $a,b\in Z_5$,则 $((a+b)\bmod 5)\in Z_5$;
            \item[结合律] $(a+b)+c \bmod 5 = a+(b+c) \bmod 5$;
            \item[单位元] $0$ 是群的单位元,$a+0\equiv a\pmod{5}$;
            \item[逆元] 每个元素都有逆元:$0+0\equiv 0\pmod{5}$;$1+4\equiv 0\pmod{5}$;$2+3\equiv 0\pmod{5}$;$3+2\equiv 0\pmod{5}$;$4+1\equiv 0\pmod{5}$。
        \end{description}
        所以 $(Z_5, +\bmod 5)$ 是一个群。
        \begin{description}
            \item[生成元] 而 $g=3$ 是它的一个生成元:

            \begin{tabular}{l|c|c|c|c|c}
                $i$            & 1 & 2 & 3 & 4 & 5 \\
                \hline
                $g^i \bmod 5$  & 3 & 1 & 4 & 2 & 0 \\
            \end{tabular}
        \end{description}
        所以 $(Z_5, +\bmod 5)$ 是一个循环群。
        \item 解:不是。$Z_8^*=\{1,3,5,7\}$,则:
        \begin{description}
            \item[封闭性] $1\times x\equiv x \pmod{8}\in Z_8^*$;$3\times 5\equiv 7\pmod{8}, 3\times 7\equiv 5\pmod{8}, 5\times 7\equiv 7\pmod{8}$;
            \item[结合律] $(a\times b)\times c\equiv a\times (b\times c)\pmod{8}$;
            \item[单位元] $1$ 是群的单位元,$a\times 1 \equiv a\pmod{8}$;
            \item[逆元] 每个元素都有逆元:$1\times 1 \equiv 1\pmod{8}, 3\times 3\equiv 1\pmod{8}, 5\times 5\equiv 1\pmod{8}, 7\times 7\equiv 1\pmod{8}$。  
        \end{description}
        所以 $(Z_8^*,\times \bmod 8)$ 是一个群。
        \begin{description}
            \item[生成元] 而它的任何一个元素都不是它的生成元:$\{1\},\{1,3\},\{1,5\},\{1,7\}$ 都是它的生成子群。
        \end{description}
        所以 $(Z_8^*,\times \bmod 8)$ 不是循环群。
    \end{enumerate}
    \item 解:由于循环群的性质:$h\circ h^{q-1}=h^q=e$,以及循环群的封闭性性质,$h^{q-1}\in G$,所以 $h^{q-1}$ 是 $h$ 的逆元。
    
    为了求出 $h^{q-1}$,可以考虑使用平方相乘法,逐步求出 $h,h^2,\cdots, h^{2^k}(k=\lfloor \log_2 (q-1) \rfloor)$,而 $q-1=(a_ka_{k-1}\cdots a_0)_2$ 表示为二进制形式,那么
    \begin{equation*}
        h^{q-1}=(h^{2^k})^{a_k}\cdot (h^{2^{k-1}})^{a_{k-1}}\cdot\cdots\cdot h^{a_0}
    \end{equation*}
    可以在 $\mathcal{O}(\log_2(q-1))$ 时间内求出。
    \item 解:$(Z_{13},+\bmod 13)$ 是满足条件的 13 阶循环群。
    \begin{description}
        \item[封闭性] 若 $a,b\in Z_{13}$,则 $((a+b)\bmod 13)\in Z_{13}$;
        \item[结合律] $(a+b)+c \bmod 13 = a+(b+c) \bmod 13$;
        \item[单位元] $0$ 是群的单位元,$a+0\equiv a\pmod{13}$;
        \item[逆元] 每个元素 $x\in Z_{13}$ 都有逆元 $(13-x)\in Z_{13}$,$x+(13-x)\equiv 0\pmod{13}$。
    \end{description}
    所以 $(Z_{13},+\bmod 13)$ 是一个群。
    \begin{description}
        \item[生成元]  $g=5$ 是它的一个生成元。
    
        \begin{tabular}{l*{13}{|c}}
        $g$          & 1 & 2  & 3 & 4 & 5  & 6 & 7 & 8 & 9 & 10 & 11 & 12 & 13\\
        \hline
        $g^i\bmod 13$ & 5 & 10 & 2 & 7 & 12 & 4 & 9 & 1 & 6 & 11 & 3  & 8  & 0 \\
        \end{tabular}
    \end{description}
    
\end{enumerate}

\end{document}